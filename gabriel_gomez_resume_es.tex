%-------------------------
% Resume in LateX
% Author : Gabriel Gomez
% License : MIT
%------------------------

\documentclass[letterpaper,11pt]{article}

\usepackage{latexsym}
\usepackage[empty]{fullpage}
\usepackage{titlesec}
\usepackage{marvosym}
\usepackage[usenames,dvipsnames]{color}
\usepackage{verbatim}
\usepackage{enumitem}
\usepackage[hidelinks]{hyperref}
\usepackage{fancyhdr}
\usepackage[spanish]{babel}
\usepackage{tabularx}
\input{glyphtounicode}

\pagestyle{fancy}
\fancyhf{} % Clear all header and footer fields
\fancyfoot{}
\renewcommand{\headrulewidth}{0pt}
\renewcommand{\footrulewidth}{0pt}

% Adjust margins
\addtolength{\oddsidemargin}{-0.5in}
\addtolength{\evensidemargin}{-0.5in}
\addtolength{\textwidth}{1in}
\addtolength{\topmargin}{-.5in}
\addtolength{\textheight}{1.0in}

\urlstyle{same}

\raggedbottom
\raggedright
\setlength{\tabcolsep}{0in}

% Sections formatting
\titleformat{\section}{
  \vspace{-4pt}\scshape\raggedright\large
}{}{0em}{}[\color{black}\titlerule \vspace{-5pt}]

% Ensure that generate PDF is machine readable/ATS parsable
\pdfgentounicode=1

%-------------------------
% Custom commands
\newcommand{\resumeItem}[2]{
  \item\small{
    \textbf{#1}{: #2 \vspace{-2pt}}
  }
}

% Just in case someone needs a heading that does not need to be in a list
\newcommand{\resumeHeading}[4]{
    \begin{tabular*}{0.99\textwidth}[t]{l@{\extracolsep{\fill}}r}
      \textbf{#1} & #2 \\
      \textit{\small#3} & \textit{\small #4} \\
    \end{tabular*}\vspace{-5pt}
}

\newcommand{\resumeSubheading}[4]{
  \vspace{-1pt}\item
    \begin{tabular*}{0.97\textwidth}[t]{l@{\extracolsep{\fill}}r}
      \textbf{#1} & #2 \\
      \textit{\small#3} & \textit{\small #4} \\
    \end{tabular*}\vspace{-5pt}
}

\newcommand{\resumeSubSubheading}[2]{
    \begin{tabular*}{0.97\textwidth}{l@{\extracolsep{\fill}}r}
      \textit{\small#1} & \textit{\small #2} \\
    \end{tabular*}\vspace{-5pt}
}

\newcommand{\resumeSubItem}[2]{\resumeItem{#1}{#2}\vspace{-4pt}}

\renewcommand{\labelitemii}{$\circ$}

\newcommand{\resumeSubHeadingListStart}{\begin{itemize}[leftmargin=*]}
\newcommand{\resumeSubHeadingListEnd}{\end{itemize}}
\newcommand{\resumeItemListStart}{\begin{itemize}}
\newcommand{\resumeItemListEnd}{\end{itemize}\vspace{-5pt}}
\setlength{\footskip}{5pt}
%-------------------------------------------
%%%%%%  CV STARTS HERE  %%%%%%%%%%%%%%%%%%%%%%%%%%%%


\begin{document}

%----------HEADING-----------------
\begin{tabular*}{\textwidth}{l@{\extracolsep{\fill}}r}
  \textbf{\href{https://gago852.github.io/}{\Large Gabriel Gomez}} & Correo electrónico: \href{mailto:ggomez.dev@outlook.com}{ggomez.dev@outlook.com}\\
  \href{https://gago852.github.io/}{gago852.github.io} & Movil: \href{tel:+573168325201}{+57-316-832-5201} \\
\end{tabular*}





%-----------EXPERIENCE-----------------
\section{Experiencia}
\resumeSubHeadingListStart

\resumeSubheading
{Extreme Technologies S.A.}{Barranquilla, CO}
{Desarrollador de Software}{Abr 2021 -- Jun 2023}
\resumeItemListStart
\resumeItem{Acuacar Daños}
{Acuacar Daños es un sistema de gestión de cuadrillas basado en la nube para empresas de servicios públicos, este sistema está hecho a medida para el control de daños en la red de agua y para la empresa de servicios públicos de agua Acuacar. Utiliza una página web para los supervisores, construida con Google Web Toolkit (GWT), y una aplicación Android utilizada por los contratistas para recibir tareas, realizar un seguimiento del progreso de la tarea utilizando datos de geolocalización (GPS) e informar al sistema.}
\resumeItem{Promigas}
{Promigas es un sistema de gestión de cuadrillas basado en la nube para empresas de servicios públicos, este sistema está hecho a medida para las inspecciones en la red de gas y para la empresa Promigas. Utiliza una página web para los supervisores, construida con Google Web Toolkit (GWT), y una aplicación para Android que se utiliza para que los contratistas reciban tareas, sigan el progreso de la tarea utilizando datos de geolocalización (GPS) e informen al sistema, también tiene una función para realizar un seguimiento de la ruta que sigue el contratista en tiempo real mientras realiza la inspección.}

\resumeItem{Podas y Lavados}
{Podas y Lavados es un sistema de gestión de cuadrillas basado en la nube para el mantenimiento de redes eléctricas y la poda de árboles, especialmente diseñado para Aire y Afinia. Utiliza una página web para los supervisores construida con Google Web Toolkit (GWT) y una aplicación Android para que los contratistas reciban tareas, sigan el progreso de la tarea utilizando datos de geolocalización (GPS) e informen al sistema. Otra función es la posibilidad de que el contratista catalogue los árboles que no están en la base de datos, que se han importado desde un archivo Keyhole Markup Language (KMZ) de Google Earth.}

\resumeItem{SEPT Santa Marta}
{SEPT Santa Marta es un sistema de transporte público en autobús basado en la nube para la ciudad de Santa Marta. Utiliza varias tecnologías, como Amazon Lambdas, Digi Routers y un validador de pasajes que funciona con Android. Otra característica es la integración con el centro de mando y control de la policía para realizar un seguimiento de los autobuses en tiempo real, el uso de un botón de pánico para llamar a la policía y el uso de monederos virtuales para pagar el billete de autobús con códigos QR.}

\resumeItemListEnd

% --------Multiple Positions Heading------------
%  \resumeSubSubheading
%   {Software Engineer I}{Oct 2014 -- Sep 2016}
%   \resumeItemListStart
%      \resumeItem{Apache Beam}
%        {Apache Beam is a unified model for defining both batch and streaming data-parallel processing pipelines}
%   \resumeItemListEnd

%-------------------------------------------

\resumeSubHeadingListEnd

%-----------EDUCATION-----------------
\section{Educación}
\resumeSubHeadingListStart
\resumeSubheading
{Universidad Autónoma del Caribe}{Barranquilla, CO}
{Pregrado en Ingeniería de sistemas}{Feb 2018 -- May 2023}
\resumeSubHeadingListEnd


%-----------PROJECTS-----------------
\section{Proyectos}
\resumeSubHeadingListStart
\resumeSubItem{Weather App}
{Aplicación meteorológica de código abierto para Android creada con Kotlin y Jetpack Compose que ofrece previsiones en tiempo real, actualizaciones basadas en la ubicación y una interfaz de usuario moderna y personalizable mediante la API OpenWeather.}
\resumeSubItem{Resume Builder on LateX}
{Un generador de currículos de código abierto que utiliza LaTeX para crear PDF de aspecto profesional y utiliza GitHub Actions para la creación y el despliegue automatizados.}
\resumeSubHeadingListEnd


%--------SKILLS------------
\section{Skills}
\resumeSubHeadingListStart
\item{
            \textbf{Lenguajes}{: Java, Kotlin, SQL, TypeScript, JavaScript, HTML, CSS, Dart, Python}
            
      }
\item {
            \textbf{Tecnologias}{: AWS, Android, React, Astro, Jetpack Compose, Flutter, PostgreSQL, MySQL, Firebase, Git, Docker, Kubernetes, Espresso, Mockito, JUnit, Test-Driven Development, Clean Architecture, SOLID, Design Patterns, RESTful APIs, Microservices, CI/CD, Agile, Scrum, Maven, Gradle, Jira}
      }
\resumeSubHeadingListEnd


%-------------------------------------------
\end{document}
